\documentclass[12pt,a4paper]{article}
\usepackage{graphicx} % Required for inserting images
\usepackage{amsmath,cite,times,enumerate,enumitem}
\usepackage[utf8]{inputenc}
\usepackage{amsfonts}
\usepackage{amssymb,pdfpages}
\usepackage{makeidx}
\usepackage{hyperref}

\title{Cognitive skills, personality, and economic preferences in
collegiate success}

\author{Anindya Singh \\ \large Shiv Nadar Institute of eminence}
\date{April 2024}

\begin{document}
\maketitle

\newpage
\begin{abstract}
    We collected multiple measures from 100 students at a small public undergraduate liberal
arts college in the Midwestern US and later assessed their academic success. The “proactive” (hard-working, persistent) aspect of the Big Five trait of Conscientiousness and not its
“inhibitive” (organized, careful) aspect is a large positive predictor for two graduation outcomes and grade point average (GPA). The Big Five trait of Agreeableness (“pro-sociality”)
is a large and negative predictor for graduation outcomes. A non-standard cognitive skill
measure (a backward-induction game) positively predicts graduation outcomes, in parallel
with its success in predicting vocational student job success (Burks et al., 2009). Patient
time preferences predict one graduation outcome and GPA
\end{abstract}
\newpage
\section{Introduction}
The formal education system plays a crucial role in modern societies as it serves as the primary means of investing in human capital. Nowadays, many occupations require individuals to have post-secondary education in order to succeed. According to conventional economic analysis, it is rational for individuals to only pursue an educational program if they have already assessed that the benefits outweigh the costs and if they are committed to completing their studies. However, there has been a growing concern regarding the high dropout rates from post-secondary educational programs in the United States. Various studies have highlighted this issue, indicating that the average retention rate across all higher education institutions is approximately 75\%, and the average graduation rate for a four-year degree within four years ranges from 35\% to 50\%. Additionally, a significant portion of students who begin a four-year degree fail to complete it. These statistics demonstrate that there are disparities in outcomes based on students' socioeconomic backgrounds. The failure to complete a degree not only impacts a student's economic prospects but also often leaves them burdened with significant debt. Given this empirical context, economists have expanded their perspective on collegiate success beyond simple models of rational choice. They now consider individual characteristics that can predict student success.\cite{article2}

In this research paper, we will analyze three different measures of college success within a sample of 100 students from the University of Minnesota, Morris (UMM), a four-year US college. Taking a broader perspective from the field of behavioral economics, we collected data in the spring of 2007 as part of a larger project. The data included information on cognitive skills such as non-verbal IQ, numeracy, and a measure of planning ability called "Hit15." Additionally, we gathered data on personality traits using the Big Five model, demographics, and economic preferences related to time and risk.

To further investigate the success of these students, we conducted a follow-up study in 2013. During this phase, we assessed three indicators of success: the final cumulative grade point average (GPA), whether students graduated within the standard four-year timeframe, and whether they graduated within six years, which serves as an alternative evaluation criterion for institutional success. We also considered cases where students did not graduate at all during the observation period.

While previous research has primarily focused on predicting grades, our main emphasis is on graduation outcomes. This is because there is a limited amount of existing work that specifically examines these important outcomes among college students.

In accordance with previous research in psychology (Chamorro-Premuzic and Furnham, 2003, 2008; Komarraju et al., 2009; Noftle and Robins, 2007; O'Connor and Paunonen, 2007), we find that the personality trait of Conscientiousness is a strong predictor of final cumulative GPA. However, as discussed in Section 3.1, this trait can be divided into two distinct "aspects": a "proactive" aspect characterized by hard work and persistence, and an "inhibitive" aspect characterized by organization and carefulness (Costa et al., 1991; DeYoung et al., 2007; Jackson et al., 1996; Roberts et al., 2005). Interestingly, our study reveals a new finding: it is only the proactive aspect of Conscientiousness that influences this relationship.

When considering each cognitive skill measure individually in a multivariate model, all of them show varying degrees of predictive power for GPA. However, when all variables are included in the model, no cognitive skill measure remains statistically significant in predicting GPA. Furthermore, our economic measures of time preferences, specifically a lower present bias and lower intensity of long-term discounting within a quasi-hyperbolic discounting framework, also predict GPA. This finding aligns with the existing research conducted by behavioral economists on time preferences (Dohmen et al., 2011; Golsteyn et al., 2014).

In terms of graduation outcomes, after adjusting for demographic variables, similar to GPA, it is evident that Conscientiousness is strongly linked to both graduation indicators. However, our findings also reveal a unique result in relation to GPA: only the "proactive" aspect, not the "inhibitive" one, influences the association with these outcomes. Moreover, in our study, this factor demonstrates a significant predictive impact (a one-standard-deviation increase is linked to a 3-fold to 7-fold rise in graduation odds.) Conversely, the personality trait Agreeableness (also discussed in Section 3.1, and essentially a measure of what experimental economists refer to as "pro-sociality") is inversely correlated with graduation outcomes. While our traditional cognitive skill assessments, IQ and numeracy, align with expectations when analyzed individually in a multivariate model, they do not hold significance in the complete model. Additionally, we observe that one of our economic indicators of time preferences (greater long-term patience) is positively associated with timely completion, but does not impact completion within six years (i.e., during the observation period).

1. The outcomes of our unique cognitive skill assessment, known as the backward induction task ("Hit15"), are quite remarkable. This task involves participants thinking in reverse from a basic numerical objective to determine the most optimal current action when engaging in a game against a computer (further elaborated in Section 4.1). We view this task as a way to gauge planning or self-organizational capabilities (Gneezy et al., 2010). Our findings indicate that this planning assessment contributes slightly to predicting timely graduation, and significantly to predicting graduation within six years (i.e. during the observation period), surpassing the predictive abilities of personality traits and our more traditional cognitive skill assessments. In fact, when comparing standardized effect sizes in regressions that include both variables, our planning assessment demonstrates a greater impact than non-verbal IQ and is more statistically significant in predicting either graduation outcome.

The discovery made on Hit15 aligns with a previous one involving a large group of adult students enrolled in a vocational training program (trainee truckers). Data was gathered using the same protocol, and these students were monitored during their employment to identify factors that influenced the successful completion of a year of service, resulting in the vocational training being free under a training contract (Hoffman and Burks, 2014). In that particular scenario, the metric also proved to be valuable in predicting the completion of training contracts, and it was interpreted as the ability to strategically plan future actions based on current circumstances and limitations (Burks et al., 2009, p. 7748). If we apply a similar interpretation to our student participants, it is possible that Hit15 measures a specific cognitive skill related to self-management that is not adequately assessed by conventional tools. Considering the established psychological evidence highlighting the significance of self-management in academic achievement (e.g. Robbins et al., 2006), it may warrant increased attention from both educational institutions and the scientific community.

The paper is structured as follows. First, we describe the location of the host institution for the study, UMM, in the higher
education industry in the US, in order to locate our subjects in the larger reference pool of US university students. Second,
we review some highlights of the relevant literature that report findings on the relationship between subject characteristics
we measure and student success. Third, we discuss the design of the data collection. Fourth, we provide a short descriptive
overview of the characteristics of the subjects, grouped by graduation success categories, and display the correlations among
our measures. Fifth, we report the results of a series of logit and Tobit regression models that explore the predictive power
of our measures for student success when we control for more than one characteristic. The final section summarizes our
results and discusses their limitations and their implications.

\section{Samples}

Much work by experimental economists and psychologists is based upon subjects drawn from the ranks of students at the colleges at which the investigators teach. This naturally raises the question of the generalizability of the findings to subject pools that are culturally, socially, or demographically different. This issue has been recently addressed by Henrich et al. (2010), who point out that the degree of concern depends on the topic of the empirical investigation. Happily, in an
analysis of college student success college students are exactly the right population. However, in order to later assess the generalizability of our findings we must still locate our subjects in a large and diverse higher education system in the US.
UMM is located on the prairie in west central Minnesota, with an enrollment of 1900 students, and a curricular design
similar to that of many private liberal arts colleges.4 Given the broadly-based character of the US higher educational system,
institutions vary in the degrees offered and their level of student selectivity, and this is true of publicly-funded institutions
such as the University of Minnesota, Morris as well as privately operated colleges and universities. The Carnegie Foundation
for theAdvancement of Teaching provides a widely-used classification systemfor US institutions of higher education that was
last updated in 2010.According to this system, UMM is classified as follows: arts and science focus, exclusively undergraduate
offering a Baccalaureate degree, full-time four-year, more selective, lower transfer-in, small, and highly residential (Carnegie
Foundation for the Advancement of Teaching, 2010). 

\section{Design and data collection}

During the spring semester of 2007 the participants took part in a series of eleven tasks that took two two-hour blocks
of time, separated by a short break. Subjects participated in a computer laboratory at UMM with dividers between the
computers, and interacted in groups of 25–35 at a time. The eight tasks providing measures in the present paper are:
(1) Personality profile.
(2) Incentivized risk aversion experiment.
(3) Demographic profile.
(4) Incentivized time preferences experiment.
(5) Incentivized non-verbal IQ test.
(6) Incentivized numeracy (quantitative literacy) test.
(7) Incentivized Hit 15 Points “backward induction” game against the computer.
(8) Survey on impatience, cooperation, etc.
More detailed descriptions of each task as well as the additional tasks completed by the subjects, are provided by Burks
et al. (2008). Here we cover a few points relevant to the present paper.
The personality instrument used was the Multidimensional Personality Questionnaire (MPQ) (Patrick et al., 2002), a
standard tool created at the University of Minnesota in 1982. The MPQ has eleven differing scales measuring primary trait dimensions. Because work in economic psychology has moved rather clearly toward a focus on the Big Five as a common
psychometric model for personality in the period since our intake data was collected (Almlund et al., 2011; Borghans et al.,
2008), we have recast our MPQ results into measurements of the Big Five personality traits, using what is known about
the mapping of MPQ scales into Big Five aspects and facets, with some assistance from questions in our impatience and
cooperation survey, following Rustichini et al. (2012). The mapping from MPQ to Big Five allows us to separate the
“proactive,” industrious, achievement-oriented aspect of Conscientiousness from the “inhibitive,” cautious, careful, and
orderly aspect, as described above in Section 3.1.
Although we have no questionnaire measure of Intellect which would be directly comparable in the format of the measurement tool to the other Big Five traits, we do have three tests measuring cognitive skills: non-verbal IQ, numeracy, and
the backward induction/planning measure, Hit15, all of which were incentivized. Non-verbal IQ was collected using a computerized adaptation of Raven’s Standard Progressive Matrices (Raven et al., 2004), an abstract reasoning task.6 Our measure
of numeracy is one half of the two-section adult test of quantitative literacy published by the Educational Testing Service
(ETS, 2014), and is designed to measure how well one can use numbers found in ads, forms, flyers, articles or other printed
materials, including being able to do appropriate arithmetic operations on the material to infer desired information (Burks
et al., 2009).7
Finally, Hit15 is a game played against the computer, once for training, and then four times for a small monetary prize; it
was developed by the investigators of Gneezy et al. (2010).
8 The goal is to be the first to add the 15th point to the points total
for the game, when each player on each turn must add 1, 2, or 3 points (adding 0 points is not permitted), and with the points
total starting out at a different level for each game. The subject and the computer take turns as to who goes first, and for
the first game the computer has the first move, and exercises a dominant strategy to win. This allows the subject to observe
the correct strategy, which has to be applied under somewhat varying circumstances (due to the alternation of turns and
changing starting points totals) to win the four games with cash payoffs. This arguably captures not just the willingness to
think ahead in the sense of setting goals, but specifically being able to reason backwards from an established goal to select
the action that is now is best to reach the goal (Gneezy et al., 2010).
We also used incentivized experiments to construct standard measures of two types of economic preferences, those for payments over time, and those for payments under uncertainty. For payments over time we offered four panels of seven questions, each offering a larger future payment versus a smaller and nearer one that decremented in \$5 jumps (e.g. \$80 in two days versus \$75 today). Two panels offered choices between today and a later time, while two offered choices between two days off and a later time. We used this instrument to construct estimates of the quasi-hyperbolic discount rates (Laibson, 1997) for each subject, estimating ˇ (Beta), the present-bias parameter, or the discount factor for now versus later, and ı (Delta), the exponential discount factor for a future date versus a later future date (following Burks et al. (2009, 2012) for vocational student subjects). For choices under uncertainty we offered four panels of six questions, each offering a choice between a dollar payment that incremented and a 50–50\% lottery that was kept constant (e.g. get \$2 for sure or take a 50–50 gamble with \$2 and \$10 payoffs); with small monetary gambles such as this,
risk neutrality is a reasonable benchmark for rational behavior. We used two of these panels (ones with all choices in the positive domain) to estimate a standard risk aversion parameter (Sigma), with risk aversion increasing in the value
of \[\sigma\]. 

\section{References}
\bibliographystyle{plain}
\bibliography{mybib}




\end{document}
